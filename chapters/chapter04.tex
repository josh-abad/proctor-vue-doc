\chapter{Results and Discussions}

This chapter will present, analyze, and interprets the data gathered and needed to implement a Web-based automated online proctor: Proctor Vue.
The quantified data are presented in tabular forms to facilitate analysis and interpretation and are arranged following the study’s objectives:

\begin{enumerate}
   \item Identify the different malpractices that is conducted during online examination.
   \item Design and create a web-based Online Examination Proctor to eliminate cheating activities during online examination.
   \item Assess the effectivity and usability of the web-based Automated Online Examination Proctor.
\end{enumerate}

\section{Malpractices during online examination}

The proponents have collected the feedback of the 18 selected respondents on their experience on current online examination system of AMA Computer College Parañaque.
They were asked about the different kinds of online cheating that they have encountered.
The response is summarized in Table 4.1.

\pagebreak

\begin{table}[h!]
   \begin{center}
      \begin{tabular}{|m{20em}|c|c|}
         \hline
         \textbf{Malpractices}                                                & \textbf{Percentage}    & \textbf{Rank} \\
         \hline
         Googling for answers                                                 & 88.88\% (16 out of 18) & 1             \\
         \hline
         Messaging someone for answers                                        & 72.22\% (13 out of 18) & 2             \\
         \hline
         Taking answer keys from other students who already finished the exam & 44.44\% (8 out of 18)  & 3             \\
         \hline
         Having someone to answer your own exam                               & 27.77\% (5 out of 18)  & 4             \\
         \hline
      \end{tabular}
   \end{center}
   \caption{Ranking of Malpractices conducted during online examination}
\end{table}

The table above stated different malpractices conducted during online examinations which is ranked according to the most and frequently used by the respondents.
\emph{Googling for answers} as the most used with (88.88\%), followed by \emph{Messaging someone for answers with} (72.22\%), \emph{Taking answer keys from other students who already finished the exam} with (44.44\%), and \emph{having someone to answer your own exam} (27.77\%).

\section{The need of Online Examination Proctor}

After the feedbacks on the encountered and experienced online cheating, the respondents then asked if they find it easy to cheat in the examination, the response is summarized in Table 4.2.

\begin{table}[h!]
   \begin{center}
      \begin{tabular}{|c|c|}
         \hline
         \multicolumn{2}{|m{20em}|}{Do you find it easy to cheat in the online examination of AMA Online Learning Program?} \\
         \hline
         Agreed    & \textbf{16}                                                                                            \\
         \hline
         Disagreed & \textbf{2}                                                                                             \\
         \hline
      \end{tabular}
   \end{center}
   \caption{Respondents’ feedback on the current system}
\end{table}

As shown in the table, 88.88\% of the respondents agreed that it is easy to cheat in the online examination of AMA Online Learning Program.

Accordingly, this result confirms that students can easily cheat if the online examination is still un-proctored.
This is relative to what is stated in \emph{The Internet and Higher Education} \cite{arnold2016cheating}, unproctored assessments is prone to academic dishonesty because of the absence of supervisions.

The results and interpretation shown above and in the previous section indicated the need of Online Examination Proctor to supervise students during their examinations.

\section{Assessment on Online Examination Proctor}

The proposed system Proctor Vue: Web-based Automated online proctor was evaluated by the same respondents in terms of Effectiveness, Functionality, and User Experience.

As shown on Table 4.3, the proposed system is evaluated by the respondents according to its functionality.
The following items included in the functionality section focuses on the ability of the features of the system to run according to its function.
Statements pertaining to the Face Detection and Alt-tab switching were evaluated and interpreted as “Agree”.

\pagebreak

\begin{table}[h!]
   \begin{center}
      \begin{tabular}{|m{20em}|c|c|}
         \hline
         \textbf{Indicator}                                                              & \textbf{$\bar{X}$} & \textbf{Verbal Interpretation} \\
         \hline
         The system detects your face easily when you uploaded your picture.             & 3.83               & Agree                          \\
         \hline
         The system can detect whether your face is seen or not while taking the exam.   & 3.61               & Agree                          \\
         \hline
         The system gives you warning when your face is not seen or detected for 10secs. & 3.78               & Agree                          \\
         \hline
         The system gives you warning when you leave your exam.                          & 3.83               & Agree                          \\
         \hline
         The system gives you warning when you stayed out of the exam for 10secs.        & 3.83               & Agree                          \\
         \hline
         The system shutdowns your exam and saved your score after receiving 5 warnings. & 3.83               & Agree                          \\
         \hline
         \textbf{Average Weighted Mean}                                                  & \textbf{3.78}      & \textbf{Agree}                 \\
         \hline
      \end{tabular}
   \end{center}
   \caption{Evaluation of the Respondents in the Proposed Web-based Automated Online Proctor According to Functionality}
\end{table}

Table 4.4 shows the results of the evaluation of the system according to Effectiveness.
This section aims to collect feedbacks on the ability of the system to eliminate cheating activities of the students.
The following data leads to the weighted mean of \textbf{3.87} which is interpreted as ‘Agree’.

\begin{table}[h!]
   \begin{center}
      \begin{tabular}{|m{20em}|c|c|}
         \hline
         \textbf{Indicator}                                                                  & \textbf{$\bar{X}$} & \textbf{Verbal Interpretation} \\
         \hline
         You cannot find answers online without receiving a warning.                         & 3.78               & Agree                          \\
         \hline
         You are the only one who can answer your own exam. As the face registered is yours. & 3.94               & Agree                          \\
         \hline
         It’s hard to cheat in this kind of online examination.                              & 3.89               & Agree                          \\
         \hline
         \textbf{Average Weighted Mean}                                                      & \textbf{3.87}      & \textbf{Agree}                 \\
         \hline
      \end{tabular}
   \end{center}
   \caption{Evaluation of the Respondents in the Proposed Web-based Automated Online Proctor According to Effectiveness}
\end{table}

Shown next is the evaluation of the proposed system according to the user’s experience (Table 4.5).
This section aims to gather data in relation to user’s experience such if the system is easy to use and if it is user friendly.
The results fall into having \textbf{3.72} as the weighted mean which is interpreted as ‘Agree’.

\begin{table}[h!]
   \begin{center}
      \begin{tabular}{|m{20em}|c|c|}
         \hline
         \textbf{Indicator}                           & \textbf{$\bar{X}$} & \textbf{Verbal Interpretation} \\
         \hline
         It is not complicated to use the application & 3.61               & Agree                          \\
         \hline
         The system is user-friendly                  & 3.83               & Agree                          \\
         \hline
         \textbf{Average Weighted Mean}               & \textbf{3.72}      & \textbf{Agree}                 \\
         \hline
      \end{tabular}
   \end{center}
   \caption{Evaluation of the Respondents in the Proposed Web-based Automated Online Proctor According to Effectiveness}
\end{table}

As shown in Table 4.6, with the summary of evaluation to the Proctor Vue: Web-based automated online proctor pursuant to Functionality, Effectiveness, and User experience, the total has an Average weighted mean of \textbf{3.79} which falls into ‘Agree’.

\begin{table}[h!]
   \begin{center}
      \begin{tabular}{|m{12em}|c|c|c|}
         \hline
         \textbf{Criteria}              & \textbf{$\bar{X}$} & \textbf{Verbal Interpretation} & \textbf{Rank} \\
         \hline
         Functionality                  & 3.78               & Agree                          & 2             \\
         \hline
         Effectiveness                  & 3.87               & Agree                          & 1             \\
         \hline
         User Experience                & 3.72               & Agree                          & 3             \\
         \hline
         \textbf{Average Weighted Mean} & \textbf{3.79}      & \textbf{Agree}                 &               \\
         \hline
      \end{tabular}
   \end{center}
   \caption{Summary of Evaluation of the Respondents in the Proposed Web-based Automated Online Proctor}
\end{table}
