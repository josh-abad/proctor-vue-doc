\chapter{Summary, Conclusion and Recommendation}

In this chapter, conclusions and recommendations will be presented in accordance to the summary and results drawn from the findings derived from gathered set of data.

\section{Summary}

With the initiative to combat online cheating for a fair online examination, the study proposed to design and develop an Online examination proctor.

The proponents used the appropriate methodology to obtain facts and information that will benefit the study.
These include the use of descriptive method and survey method.
The survey method was used to recognize the cheating activities that are performed during an Unproctored online examination.
The data gathered are used as the basis of the features included in the proposed system.
With the information obtained from several literature and studies, the proponents then used it to prepare the requirements needed to the further development of the proposed system.
The analytical tools, different diagrams, software design, and prototype were created to help the proponents visualize the build of the system and its functions.

The proponents used Purposive Sampling as the sampling technique, and selected 18 students who is currently enrolled in the Full online learning platform of AMA Computer Parañaque.
The proposed study “Proctor Vue: Web-based Automated online proctor” was evaluated in terms of the following:

\textbf{Functionality}.
the proposed system is evaluated by the respondents by its functionality.
This section provides data gathered in relation to the features of the application that is working according to its use.
The evaluation scored an Average weighted mean of \textbf{3.78} which is interpreted as ‘Agree’.
This indicates that the proposed system is agreed to be functioning properly.

\textbf{Effectiveness}.
This section focuses on the effectivity of the system to combat online cheating that is happening in the unproctored online examination.
With the Average Weighted mean of \textbf{3.87} the interpretation leads to ‘agree’ which means that the system has the potential to aid online cheating.

\textbf{User Experience}.
the proposed system is evaluated by the respondents by its user experience.
This section gathered insights about the smoothness of the system to provide a good experience to the users.
The evaluation scored \textbf{3.72} as the weighted mean which is ‘Agree’ that indicates that the proposed system is user—friendly according to the results.

\section{Conclusion}

Drawn from the study are the following conclusions:

\begin{enumerate}
   \item Most of the students of AMA Computer College Parañaque has encountered cheating during the current online examination of AMA.
   \item Most of the students find it easy to cheat online since the examination is unproctored.
   \item Most of the answers in the examination is uploaded online and is easy to access and use.
   \item Students can let someone answer their own exam since exams are not proctored.
   \item The procedures and methods on the Web-based Automated Online Examination Proctor were evaluated “Agree”.
\end{enumerate}

\section{Recommendation}

Derived from the findings and conclusions of the study is the following recommendations:

\begin{enumerate}
   \item Online Examination Proctor is needed to prevent online cheating during online examinations.
   \item Future Researchers may explore other types of Online examination proctor to provide equivalent performance.
   \item Future Researchers may work on another features that will provide a huge impact in eliminating cheating activities.
   \item Future Researchers should use a computer with high specs specially the RAM, Processor and Graphics Card for modelling 3D figures and developing the application.
\end{enumerate}
